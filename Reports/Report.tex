\documentclass[12pt,a4paper]{article}
% \usepackage[english]{babel}
% \usepackage[utf8x]{inputenc}

\usepackage{graphicx} % Required for inserting images.
\usepackage[margin=25mm]{geometry}
\parskip 4.2pt  % Sets spacing between paragraphs.
% \renewcommand{\baselinestretch}{1.5}  % Uncomment for 1.5 spacing between lines.
\parindent 8.4pt  % Sets leading space for paragraphs.
\usepackage[font=sf]{caption} % Changes font of captions.

\usepackage{amsmath}
\usepackage{amsfonts}
\usepackage{amssymb}
\usepackage{siunitx}
\usepackage{verbatim}
\usepackage{hyperref} % Required for inserting clickable links.
\usepackage{natbib} % Required for APA-style citations.


\title{Do cryptocurrencies extend the mean-variance frontier of an equity investor?}
\author{Sander Naerum, Thomas Pietsch and Ziga Jagodnik}

\begin{document}
\maketitle

\begin{abstract}
In this project we will investigate if cryptocurrencies extend the mean-variance frontier of an equity investor. 
By using an industry portfolio dataset consisting of 12 different industries collected from Kenneth French website 
combined with “” cryptocurrency index, we extract the mean-variance frontier. We show that adding cryptocurrencies to 
the mean-variance frontier do not have a significant impact.  
\end{abstract}

\section{Introduction}\label{sec:intro}
The mean-variance frontier is a mathematical framework for building portfolios that aim to maximize expected returns 
while controlling a set level of risk. This concept extends diversification in investing, highlighting that owning 
diverse financial assets is less risky than focusing on a single asset class. The key idea is that an asset's risk and 
return should be evaluated in the context of its contribution to the overall risk and return of a diversified portfolio. 
Historical asset price variance is used as a proxy for estimating future risk in the mean-variance frontier.

In the evolving landscape of financial assets, the integration of cryptocurrencies adds a new dimension to portfolio 
construction. Cryptocurrencies, such as Bitcoin and Ethereum, bring unique characteristics and opportunities to the mix. 
Therefore, it is interesting to explore if this new universe of assets can enhance portfolio diversification. 

\section{Methodology}\label{sec:methods}
\subsection{Assumptions}
\subsection{Return calculations}
The data downloaded from Kenneth French's website is already calculated as simple returns, however, in this paper we will use the log returns of the assets. To convert the simple returns to log, we use the following formula: 
$$r_i = ln(1+r_i)$$

\noindent For the data on the crypto prices downloaded from yahoo finance we use the following formula to obtain the log returns: 
$$r_i = ln(\frac{P_t}{P_t-1})$$

\section{Data}\label{sec:result}
\textbf{Equity data:} Kenneth French website, 12 industry portfolios. 

\noindent\textbf{Crypto's:} Imported the 5 largest crypto currencies based on market cap, source: Yahoo Finance.  

\section{Results}\label{sec:result}

\section{Discussion}\label{sec:discussion}

Blah

\subsection{Limitations}\label{sec:limitations}

Blah blah

\subsection{Future work}\label{sec:future}

Did you see what I wrote in Sec.~\ref{sec:intro}? Not too controversial I hope!

\begin{figure}[htbp!]
\begin{center}
\includegraphics[width=0.5\columnwidth]{venn_discoveries.pdf}
\end{center}
\caption{Yes, put a few words or sentences here explaining what is in the figure.}
\label{fig:venn}
\end{figure}

\bibliographystyle{apalike}
\bibliography{example}

\end{document}
